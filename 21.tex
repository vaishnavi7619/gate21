\let\negmedspace\undefined
\let\negthickspace\undefined
\documentclass[a4,12pt,onecolumn]{IEEEtran}
\usepackage{amsmath,amssymb,amsfonts,amsthm}
\usepackage{algorithmic}
\usepackage{graphicx}
\usepackage{textcomp}
\usepackage{xcolor}
\usepackage{txfonts}
\usepackage{listings}
\usepackage{enumitem}
\usepackage{mathtools}
\usepackage{gensymb}
\usepackage[breaklinks=true]{hyperref}
\usepackage{tkz-euclide}
\usepackage{listings}
\usepackage{circuitikz}
\begin{document}

\title{ signals and systems
\vspace{1cm}

Gate2021-ec-Q52}
\author{EE23BTECH11014- Devarakonda Guna vaishnavi}
\maketitle
\textbf{Question:}

A message signal having peak-to-peak value of $2 \, \text{V}$, root mean square value of $0.1 \, \text{V}$, and bandwidth of $5 \, \text{kHz}$ is sampled and fed to a pulse code modulation (PCM) system that uses a uniform quantizer. The PCM output is transmitted over a channel that can support a maximum transmission rate of $50 \, \text{kbps}$. Assuming that the quantization error is uniformly distributed, calculate the maximum signal-to-quantization noise ratio (rounded off to two decimal places).

solution:
\begin{table}[h!]
    \centering
    \begin{center}
    \begin{tabular}{|c|p{8cm}|}
    \hline
    \textbf{term} & \textbf{Description} \\
    \hline
    Peak-to-Peak & Peak-to-peak value of the message signal (in volts). \\
    \hline
    RMS & Root mean square value of the message signal (in volts). \\
    \hline
    Bandwidth & Bandwidth of the message signal (in kilohertz). \\
    \hline
    N & Number of bits per sample used in quantization. \\
    \hline
    Quantization range & Range of values in which the quantized signal falls (in volts). \\
    \hline
    Step size & The size of each quantization interval (in volts). \\
    \hline
    Sampling rate & Rate at which the message signal is sampled (in kilohertz). \\
    \hline
    Maximum transmission rate & Maximum transmission rate supported by the channel (in kilobits per second). \\
    \hline
    SQNR & Signal-to-quantization noise ratio. \\
    \hline
    \end{tabular}
\end{center}

    \caption{Input Parameters}
    \label{table:parameters}
\end{table}


\begin{enumerate}
    
    \item \textbf{Calculate SQNR:}
    \begin{equation}\label{eq:sqnr}
        \text{SQNR} = 6.02 \times N + 1.76
         \end{equation}
\end{enumerate}




 $\implies \text{SQNR} = 31.86$

 






\end{document}

